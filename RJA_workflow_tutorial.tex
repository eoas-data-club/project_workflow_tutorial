\documentclass[]{article}
\usepackage{lmodern}
\usepackage{amssymb,amsmath}
\usepackage{ifxetex,ifluatex}
\usepackage{fixltx2e} % provides \textsubscript
\ifnum 0\ifxetex 1\fi\ifluatex 1\fi=0 % if pdftex
  \usepackage[T1]{fontenc}
  \usepackage[utf8]{inputenc}
\else % if luatex or xelatex
  \ifxetex
    \usepackage{mathspec}
  \else
    \usepackage{fontspec}
  \fi
  \defaultfontfeatures{Ligatures=TeX,Scale=MatchLowercase}
\fi
% use upquote if available, for straight quotes in verbatim environments
\IfFileExists{upquote.sty}{\usepackage{upquote}}{}
% use microtype if available
\IfFileExists{microtype.sty}{%
\usepackage{microtype}
\UseMicrotypeSet[protrusion]{basicmath} % disable protrusion for tt fonts
}{}
\usepackage[margin=1in]{geometry}
\usepackage{hyperref}
\hypersetup{unicode=true,
            pdftitle={Project Workflow Tutorial},
            pdfborder={0 0 0},
            breaklinks=true}
\urlstyle{same}  % don't use monospace font for urls
\usepackage{color}
\usepackage{fancyvrb}
\newcommand{\VerbBar}{|}
\newcommand{\VERB}{\Verb[commandchars=\\\{\}]}
\DefineVerbatimEnvironment{Highlighting}{Verbatim}{commandchars=\\\{\}}
% Add ',fontsize=\small' for more characters per line
\usepackage{framed}
\definecolor{shadecolor}{RGB}{248,248,248}
\newenvironment{Shaded}{\begin{snugshade}}{\end{snugshade}}
\newcommand{\AlertTok}[1]{\textcolor[rgb]{0.94,0.16,0.16}{#1}}
\newcommand{\AnnotationTok}[1]{\textcolor[rgb]{0.56,0.35,0.01}{\textbf{\textit{#1}}}}
\newcommand{\AttributeTok}[1]{\textcolor[rgb]{0.77,0.63,0.00}{#1}}
\newcommand{\BaseNTok}[1]{\textcolor[rgb]{0.00,0.00,0.81}{#1}}
\newcommand{\BuiltInTok}[1]{#1}
\newcommand{\CharTok}[1]{\textcolor[rgb]{0.31,0.60,0.02}{#1}}
\newcommand{\CommentTok}[1]{\textcolor[rgb]{0.56,0.35,0.01}{\textit{#1}}}
\newcommand{\CommentVarTok}[1]{\textcolor[rgb]{0.56,0.35,0.01}{\textbf{\textit{#1}}}}
\newcommand{\ConstantTok}[1]{\textcolor[rgb]{0.00,0.00,0.00}{#1}}
\newcommand{\ControlFlowTok}[1]{\textcolor[rgb]{0.13,0.29,0.53}{\textbf{#1}}}
\newcommand{\DataTypeTok}[1]{\textcolor[rgb]{0.13,0.29,0.53}{#1}}
\newcommand{\DecValTok}[1]{\textcolor[rgb]{0.00,0.00,0.81}{#1}}
\newcommand{\DocumentationTok}[1]{\textcolor[rgb]{0.56,0.35,0.01}{\textbf{\textit{#1}}}}
\newcommand{\ErrorTok}[1]{\textcolor[rgb]{0.64,0.00,0.00}{\textbf{#1}}}
\newcommand{\ExtensionTok}[1]{#1}
\newcommand{\FloatTok}[1]{\textcolor[rgb]{0.00,0.00,0.81}{#1}}
\newcommand{\FunctionTok}[1]{\textcolor[rgb]{0.00,0.00,0.00}{#1}}
\newcommand{\ImportTok}[1]{#1}
\newcommand{\InformationTok}[1]{\textcolor[rgb]{0.56,0.35,0.01}{\textbf{\textit{#1}}}}
\newcommand{\KeywordTok}[1]{\textcolor[rgb]{0.13,0.29,0.53}{\textbf{#1}}}
\newcommand{\NormalTok}[1]{#1}
\newcommand{\OperatorTok}[1]{\textcolor[rgb]{0.81,0.36,0.00}{\textbf{#1}}}
\newcommand{\OtherTok}[1]{\textcolor[rgb]{0.56,0.35,0.01}{#1}}
\newcommand{\PreprocessorTok}[1]{\textcolor[rgb]{0.56,0.35,0.01}{\textit{#1}}}
\newcommand{\RegionMarkerTok}[1]{#1}
\newcommand{\SpecialCharTok}[1]{\textcolor[rgb]{0.00,0.00,0.00}{#1}}
\newcommand{\SpecialStringTok}[1]{\textcolor[rgb]{0.31,0.60,0.02}{#1}}
\newcommand{\StringTok}[1]{\textcolor[rgb]{0.31,0.60,0.02}{#1}}
\newcommand{\VariableTok}[1]{\textcolor[rgb]{0.00,0.00,0.00}{#1}}
\newcommand{\VerbatimStringTok}[1]{\textcolor[rgb]{0.31,0.60,0.02}{#1}}
\newcommand{\WarningTok}[1]{\textcolor[rgb]{0.56,0.35,0.01}{\textbf{\textit{#1}}}}
\usepackage{graphicx,grffile}
\makeatletter
\def\maxwidth{\ifdim\Gin@nat@width>\linewidth\linewidth\else\Gin@nat@width\fi}
\def\maxheight{\ifdim\Gin@nat@height>\textheight\textheight\else\Gin@nat@height\fi}
\makeatother
% Scale images if necessary, so that they will not overflow the page
% margins by default, and it is still possible to overwrite the defaults
% using explicit options in \includegraphics[width, height, ...]{}
\setkeys{Gin}{width=\maxwidth,height=\maxheight,keepaspectratio}
\IfFileExists{parskip.sty}{%
\usepackage{parskip}
}{% else
\setlength{\parindent}{0pt}
\setlength{\parskip}{6pt plus 2pt minus 1pt}
}
\setlength{\emergencystretch}{3em}  % prevent overfull lines
\providecommand{\tightlist}{%
  \setlength{\itemsep}{0pt}\setlength{\parskip}{0pt}}
\setcounter{secnumdepth}{0}
% Redefines (sub)paragraphs to behave more like sections
\ifx\paragraph\undefined\else
\let\oldparagraph\paragraph
\renewcommand{\paragraph}[1]{\oldparagraph{#1}\mbox{}}
\fi
\ifx\subparagraph\undefined\else
\let\oldsubparagraph\subparagraph
\renewcommand{\subparagraph}[1]{\oldsubparagraph{#1}\mbox{}}
\fi

%%% Use protect on footnotes to avoid problems with footnotes in titles
\let\rmarkdownfootnote\footnote%
\def\footnote{\protect\rmarkdownfootnote}

%%% Change title format to be more compact
\usepackage{titling}

% Create subtitle command for use in maketitle
\providecommand{\subtitle}[1]{
  \posttitle{
    \begin{center}\large#1\end{center}
    }
}

\setlength{\droptitle}{-2em}

  \title{Project Workflow Tutorial}
    \pretitle{\vspace{\droptitle}\centering\huge}
  \posttitle{\par}
    \author{}
    \preauthor{}\postauthor{}
    \date{}
    \predate{}\postdate{}
  

\begin{document}
\maketitle

\#hi hello

\hypertarget{hi-all-to-get-started-please-do-the-following-steps}{%
\subparagraph{Hi all! To get started, please do the following
steps:}\label{hi-all-to-get-started-please-do-the-following-steps}}

\begin{enumerate}
\def\labelenumi{\arabic{enumi}.}
\tightlist
\item
  Clone the ``project\_workflow\_tutorial'' repository (hit the green
  ``Clone or Download'' button and then copy the URL)
\item
  Open RStudio and go to File \textgreater{} New Project. Choose the
  option to check out a project from version control, and paste in the
  repository URL
\item
  Pull the latest version of the repository from Github (good practice
  for later)
\item
  Open the ``project\_workflow\_tutorial.Rmd'' script
\item
  Save the RMarkdown script as a new file, adding your initials or a
  team name. This will allow us to work on the same script in the same
  repository without creating commit conflicts, so it's important to do
  first, before making other changes!
\item
  Check that you have all the packages we're going to be using installed
  in your version of RStudio.
\item
  Push your changes to Github
\end{enumerate}

\hypertarget{background}{%
\subsubsection{Background}\label{background}}

In this tutorial, we'll be using a simplified version of the data I
collected this summer in Mongolia. The project was designed to study the
effect of turbidity and other environmental variables on fishes'
vulnerability to fishing gear. Mongolia's rivers are becoming more
turbid because of thunderstorms caused by climate change, and we want to
understand how this might affect fish behavior and fishing success.

We hypothesize that high turbidity reduces catch rates. But there's lots
of evidence that other environmental variables affect catch rate,
including weather (like degree of cloud cover), barometric pressure,
water temperature, and time of day. Unless we control for these
variables, we can't be sure that we're measuring the effect of turbidity
without bias. There are additional complexities having to do with the
format of the data that will also affect the structure of our
statistical model. We'll get to these shortly.

\hypertarget{step-1-setup}{%
\subsubsection{Step 1: Setup}\label{step-1-setup}}

Load the following libraries and use read.csv() to read the
``catchrates.csv'' file into R.\\
Note: because you've downloaded the .Rproj file to your local
repository, which means everything in this folder belongs to the same R
project, you shouldn't have to give read.csv a file path for the catch
rates data. RStudio will just look for a file named ``catchrates.csv''
within the project folder.

\begin{Shaded}
\begin{Highlighting}[]
\KeywordTok{rm}\NormalTok{(}\DataTypeTok{list =} \KeywordTok{ls}\NormalTok{())       }\CommentTok{# clear your workspace}

\KeywordTok{library}\NormalTok{(tidyverse)    }\CommentTok{# for manipulating and plotting data}
\KeywordTok{library}\NormalTok{(ggbiplot)     }\CommentTok{# for plotting a PCA}
\KeywordTok{library}\NormalTok{(lme4)         }\CommentTok{# to fit a generalized linear mixed model}
\KeywordTok{library}\NormalTok{(lubridate)    }\CommentTok{# to work with date and time objects}
\KeywordTok{library}\NormalTok{(reshape2)     }\CommentTok{# to reshape data}
\KeywordTok{library}\NormalTok{(MuMIn)        }\CommentTok{# for model validation}
\KeywordTok{library}\NormalTok{(DHARMa)       }\CommentTok{# for model validation}
\KeywordTok{library}\NormalTok{(effects)      }\CommentTok{# to plot partial effects}

\KeywordTok{options}\NormalTok{(}\DataTypeTok{scipen=}\DecValTok{999}\NormalTok{)   }\CommentTok{# turn off scientific notation}
\end{Highlighting}
\end{Shaded}

\hypertarget{step-2-what-is-the-distribution-of-the-response-variable-catch-rate}{%
\subsubsection{Step 2: What is the distribution of the response variable
catch
rate?}\label{step-2-what-is-the-distribution-of-the-response-variable-catch-rate}}

\textbf{Use the function hist() to visualize this}

\begin{Shaded}
\begin{Highlighting}[]
\NormalTok{data <-}\StringTok{ }\KeywordTok{read.csv}\NormalTok{(}\StringTok{"catchrates.csv"}\NormalTok{) }
\NormalTok{catchrate_hist <-}\StringTok{ }\KeywordTok{hist}\NormalTok{(data}\OperatorTok{$}\NormalTok{catchrate) }\CommentTok{#add hist and trying again}
\end{Highlighting}
\end{Shaded}

\includegraphics{RJA_workflow_tutorial_files/figure-latex/histogram-1.pdf}

\hypertarget{follow-up-questions}{%
\paragraph{Follow-up questions:}\label{follow-up-questions}}

\begin{enumerate}
\def\labelenumi{\arabic{enumi}.}
\tightlist
\item
  Why is checking the distribution of the response variable important?
\item
  What does it tell us about how to structure the statistical model we
  want to fit to these data?
\end{enumerate}

\hypertarget{step-3-can-we-see-a-relationship-between-turbidity-and-catch-rate-visually-without-using-any-stats}{%
\subsubsection{Step 3: Can we see a relationship between turbidity and
catch rate visually, without using any
stats?}\label{step-3-can-we-see-a-relationship-between-turbidity-and-catch-rate-visually-without-using-any-stats}}

\textbf{Use the function plot()}

\hypertarget{step-4-plot-the-relationship-between-other-potential-explanatory-variables-weather-barometric-pressure-water-temperature-air-temperature-time-of-day-etc.-and-turbidity.}{%
\subsubsection{Step 4: Plot the relationship between other potential
explanatory variables (weather, barometric pressure, water temperature,
air temperature, time of day, etc.) and
turbidity.}\label{step-4-plot-the-relationship-between-other-potential-explanatory-variables-weather-barometric-pressure-water-temperature-air-temperature-time-of-day-etc.-and-turbidity.}}

\textbf{Use par(mfrow = c(\#rows, \#cols) to make a multipanel plot in
base R. Use plot() for continuous variables and boxplot() for
categorical variables.}

Note: All the variables in columns 3-13 are fair game! But working with
dates and times in R can be complex, so to keep us moving through the
material, let's set them aside for later unless you're really
comfortable with the 'lubridate"'package.

\hypertarget{follow-up-question}{%
\paragraph{Follow-up question:}\label{follow-up-question}}

\begin{enumerate}
\def\labelenumi{\arabic{enumi}.}
\tightlist
\item
  Based on this exploratory plotting, what covariates do you want to
  include in your model?
\end{enumerate}

\hypertarget{step-5-assessing-collinearity-between-explanatory-variables}{%
\paragraph{Step 5: Assessing collinearity between explanatory
variables}\label{step-5-assessing-collinearity-between-explanatory-variables}}

If you include variables that are too closely correlated with each
other, the model fitting formula you use is going to have trouble.

\textbf{Use the cor() function and ggheatmap() to make a correlation
coefficient matrix to see which of your explanatory variables are
strongly related to each other and to your variable of interest,
turbidity.}

Note: this is a good opportunity to practice finding ready-made code on
the internet. Making correlation matrices is something that scientists
have to do a lot, so why reinvent the wheel?

\hypertarget{follow-up-questions-1}{%
\paragraph{Follow up questions:}\label{follow-up-questions-1}}

\begin{enumerate}
\def\labelenumi{\arabic{enumi}.}
\tightlist
\item
  Why are highly correlated variables a problem, statistically speaking?
\item
  How do you interpret a correlation coefficient? What's the threshold
  for significance?
\item
  Let's interpret our correlation coefficient plots
\end{enumerate}

\hypertarget{step-6-principal-components-analysis-of-explanatory-variables}{%
\subsubsection{Step 6: Principal Components Analysis of explanatory
variables}\label{step-6-principal-components-analysis-of-explanatory-variables}}

We've got a number of variables that we need to control for, but we're
not primarily interested in them. Several of them are correlated with
each other, so we probably want to find a way to figure out which ones
give us essentially the same information and collapse them into a single
``environmental profile'' of each observation. One common way to do this
is Principal Components Analysis (PCA).

\textbf{Standardize your variables using scale(), then use the function
prcomp() to run a PCA on the covariates that you think are likely to be
a) interesting and b) related to each other. Plot your PCA output using
ggbiplot(). If you think one principal component stands out as
summarizing your environmental variables well, extract its values for
our data and add them to our dataframe to use as an explanatory
variable.}

Notes: 1. PCA only works for continuous variables 2. ggbiplot() allows
you to group the points on your graph based on categorical values. Use
the ``groups'' argument in ggbiplot to see whether the spread of your
environmental variables seems to be linked to the locations where the
data were gathered (coded in the nested variables ``camp'' and ``site'')

\hypertarget{follow-up-questions-2}{%
\paragraph{Follow-up questions}\label{follow-up-questions-2}}

\begin{enumerate}
\def\labelenumi{\arabic{enumi}.}
\tightlist
\item
  Why is it helpful to standardize our explanatory variables?
\item
  Let's summarize what we know about the data so far.
\end{enumerate}

\hypertarget{step-7-fit-a-generalized-linear-mixed-model-to-the-data}{%
\subsubsection{Step 7: Fit a generalized linear mixed model to the
data}\label{step-7-fit-a-generalized-linear-mixed-model-to-the-data}}

\textbf{Fit a generalized linear mixed effects model to these data with
a Poisson distribution and a log link and site as a random effect using
the glmer() function.}

\hypertarget{step-8-evaluate-your-models-fit-using-summary-r.squaredglmm-in-the-mumln-package-and-the-dharma-package.-plot-partial-effects-with-the-effects-package.}{%
\subsubsection{\texorpdfstring{Step 8: \textbf{Evaluate your model's fit
using summary(), r.squaredGLMM() in the `MuMln' package, and the
`DHARMa' package. Plot partial effects with the `effects'
package.}}{Step 8: Evaluate your model's fit using summary(), r.squaredGLMM() in the `MuMln' package, and the `DHARMa' package. Plot partial effects with the `effects' package.}}\label{step-8-evaluate-your-models-fit-using-summary-r.squaredglmm-in-the-mumln-package-and-the-dharma-package.-plot-partial-effects-with-the-effects-package.}}


\end{document}
